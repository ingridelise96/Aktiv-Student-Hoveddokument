\cleardoublepage
\chapter{Diskusjon}
\label{chap:discussion} 

I dette kapittelet vil resultatet bli diskutert. Kapittelet vil fokusere på om resultatet ble som forventet, om oppdragsgiver var fornøyd med resultatet, om gruppen kunne ha gjort noe annerledes/bedre og hva gruppen har lært underveis. Det vil også inneholde en vurdering av tjenesten Aktiv Student opp mot relatert arbeid. Gruppen vil gi anbefalinger og fremlegge designprinsipper for videre arbeid og utvikling av tjenesten i praksis. Kapitellet avsluttes med en konklusjon med en oppsummering av arbeidet og prosessen.

\section{Resultat i forhold til prosjektbeskrivelse}
Notater:
Ble ikke utviklet en funksjon for å hente inn informasjon om organisasjoner.
La til sosialt aspekt/kommunikasjon mellom brukere.
La til kartleggingstest/forslag til aktiviteter for brukere.
Har laget noe mer enn det prosjektbeskrivelsen foreslo.

\section{Gruppens evaluering}

gjennomførelse 

det å disponere egen tid opp imot en så stor oppgave har hvert en verdifull øvelse for oss som gruppe, og noe helt nytt for oss å måtte sette seg inn i. På bakgrunn av det vi har vært igjennom føler vi at vi har løst dette bra og fått et svært godt resultat. vi som gruppe føler at vi har lært hvordan en ordentlig arbeidsdag fungerer og hvor mye verdi det å ha struktur i en arbeidsprosess har å si. Vi har lært hvor viktig det er å sette krav til hverandre for å få en slik prosess i mål og viktigheten med å ha et produkt man er stolt av og har et forhold til.
Det å kjenne på skuffelsen når ting ikke går etter planen eller man støter på utfordringer har hvert en del av oppgaven, man at vi sammen har fått muligheten til å skape/gjøre noe unikt har økt og opprettholdt motivasjonen vår gjennom prosjektet.

gruppearbeidet

Gruppearbeidet gjennom denne bachelor oppgaven har hatt sine nedturer men mest oppturer. Et slikt arbeid som krevar at vi jobber parallelt har vært vanskelig til tider hvor man som enkelt person kan blir usikker på hva som er gjort og det at man må stole på at alle medlemmer av gruppen gjør det de blir satt til. itillegg blir det vanskelig og jobbe ut fra og stole på at den informasjonen andre har skrevet er korrekt. Vi som gruppe hadde en periode rett etter Covid-19 utbruddet som gjorde at det ble lite arbeid gjort ettersom de daglige rutinene for alle medlemmer ble endret såpass drastisk. Ut ifra dette utbruddet og endringen som ble gjort bestemte vi oss for å ha et møte hvor vi la frem for hverandre hvordan vi så for oss veien videre og satte krav til hverandre. Dette møtet kom med mye bra for vårt gruppearbeid hvor vi bestemte oss for å ha daglige møter over discord hvor hvert enkelt medlem skulle presentere hva som var blitt gjort dagen før og fordele nye arbeidsoppgaver oss imellom. Denne typen måte å arbeide på gjorde arbeidsprosessen vår veldig mye bedre og skapte en informasjonsflyt som gjorde at alle medlemene fikk god kontoll rundt hva som foregikk i prosjektet til en hver tid. Vi alle fikk et bedre forhold til oppgaven og klarte med denne løsningen å spille på styrkene til hverandre for å løse problemer og utforme de forskjellige delene av prosjektet. Covid-19 
gjorde så vi måtte jobbe hjemmeifra som førte til at alle fikk mer tid til å arbeide ettersom det fjernet reisetid for alle medlemmer og førte til at mange av våre personlige forpliktelser forsvant fra dagliglivet.

Vi har rådført oss med veileder angående formulering, oppdeling, mangler og endringer som må gjøres i hoveddokumentet. Gruppen har hatt et gått arbeidsforhold med veilederen hvor vi har fått jobbe selvstendig og fått kommet til veileder når vi trengte hjelp. Som gruppe kunne vi ha brukt veileder oftere tidlig i bacheloren, men siden vi ikke brukte veilederen så ofte i starten så føler vi at bacheloren vår viser selvstendig arbeid. Det ble mer kontakt med veileder når det nærmet seg slutten av bacheloren når vi trengte mer hjelp til strukturering hoveddokumentet. 

Når det kommer til arbeidsforholdet til oppdragsgiver har så har vi fått mye frihet rundt oppdraget. Vi har vist frem alt som har blitt gjort og sjekket om oppdragsgiver liker det som er gjort eller om han ville at at vi skulle gjøre noen endringer eller forbedringer. Hver gang vi hadde møter sjekket vi også om det var noe som han ville ha med eller om det var noe vi kunne prøve på. 

Dokumentasjon og prototypen 

Noe vi er veldig takknemlig for er mengden frihet vi har fått av oppdragsgiver som har latt oss jobbe fritt og kreativt. Dette har latt oss utforske og oppdage hvordan det er vanskelig å legge fra seg forutsetninger som man starter med i et prosjekt og ideer som ikke fungerer til fordel for det beste for oppgaven. På grunn av endringene som har blitt gjort har det vært vanskelig for oss å se enden innimellom gjennom prosjektperioden og det å finne et naturlig endepunkt på prosessen. men ut ifra tips fra både veileder og oppdragsgiver føler vi at vi fant den mest naturlige enden på vårt bidrag opp mot denne tjenesten.

Noen av utfordringene vi har støtt på med dokumentasjon og selve hoveddokumentet har hvert viktigheten med å dokumentere i sanntid ved brukerundersøkelser som vi oppdaget på noen av de initielle brukerintervjuene. vi har fått erfare hvordan det er når planen som er lagt blir endret iforhold til unntakstilstander Norge og når start planen ikke helt går som det skal på grunn av at prosjektbeskrivelsen måtte endres for å skape en bedre tjeneste. Gjennom oppgaven har vi blitt introdusert til det å skrive en stor akademisk tekst som har bydd på en bratt læringskurve. Vi har spesielt fått kjenne på hvor vanskelig det er å skrive på en akademisk måte og måtte dokumentere alt samtidig som dette skal følge metodebruk, kildehenvisning og konsistent og riktig ordlegging. Men når alt kommer til alt er vi veldig fornøyde og stolte over det vi har produsert og ser på det som en veldig god erfaring å ha hatt mulighetet til å skrive en slik oppgave. Hele denne oppgaven og prosessen vi har vært gjennom har vært en super måte å forberede seg på et arbeidsliv som vil komme med arbeidsforhold til andre, profesjonalitet i arbeidet, komme seg ut av komfortsonen og løse problemer man kommer til å støte på.


\section{Erfaringer gjort underveis}
Notater:
Digitalt gruppearbeid.
Gjennomføring av brukerundersøkelser i et større prosjekt.
Oppsett og struktur på større faglige oppgaver.
Tverrfaglig prosjektarbeid.
Verdien av å konsultere fagpersoner og eksperter, samt hvordan finne disse.
Hvordan forholde oss til og samarbeide med en oppdragsgiver.
Finne løsninger som fungerer som en middelvei mellom forskjellige ønsker.
Systematisk problemløsing.
Selvstendig planlegging og utførelse av en brukerorientert designprosess.
Akademisk skriving.
Selvstendig utforming av krav til tjenesten Aktiv Student.

Vi har gjort en del erfaringer underveis i dette bachelor prosjektet. Noe som vi har måtte lært oss på grunn av Covid-19 er digitalt gruppearbeid, og det å gjennomføre mye av prosjektet selvstendig. Spesielt rundt dette med oppsett og struktur i større akademiske dokumenter, selvstendig planlegging og utførelse av en brukerorientert prosess og selvstendig utforming av krav for et nettbasert system. Noe annet vi har utviklet av personlige egenskaper gjennom denne oppgaven er gruppearbeid i et større prosjekt, akademisk skriving og dokumentasjon av dette, hvordan forholde oss til en arbeidsgiver/oppdragsgiver og finne frem til felles løsninger som passer for alle parter, problemløsning og ikke minst tverrfaglig arbeid.

\subsection{Hva kunne blitt gjort annerledes?}
Notater: presentere 3 ulike løsninger til oppdragsgiver, utforsket aspektet om drift av selve tjenesten, tekniske løsninger for drift, prøvd å ha flere deltakere for brukerundersøkelser, tatt kontakt med komunner og frivillighetssentralen tidligere, tatt kontakt med Anders Midtsundstad tidligere for slutt evaluering av konseptet.
Lagt en konkret plan for formatering av dokumentet.
Brukt mer tid på kilder og research.

Noe som vi kunne gjort annerledes var at når vi kom til en problemstilling så ble dette løst internt i prosjektgruppen, men oppdragsgiver kunne heller blitt inkludere oppdragsgiver litt mer i denne prosessen ved å for eksempel komme opp med tre løsningsforslag og presentert dette for han med ulemper og fordeler. Hvis oppdragsgiver hadde hatt bistand og investering i prosjektet kunne vi også fremlagt løsningen med uttenkte kostnader og analyser bundet opp mot hvert enkelt løsningsforslag. Vi som gruppe skulle også ønske at vi brukte litt mer tid i starten av oppgaven på å forme skrivet litt mer akademisk ved å sette opp en bedre og konkret plan for formatering av dokumentet, og brukt litt mer tid på kilder og research. En annen ting prosjekt gruppen kunne gjort annerledes var å fokusere litt mer på tekniske løsningsforslag og løsning for drifting av tjenesten. Gruppen har gjennom hele prosjektet ønsket flere deltakere på våre undersøkelser, noe som har blitt vanskelig på grunn av Covid-19, men samtidig ser gruppen at dette kunne vært jobbet litt hardere for å få mulig. Til slutt er det et par ting som prosjektgruppen ønsker de tenkte på før som var det å kontakte komunner og frivillighetssentralen tidligere og det å få en slutt vurdering av konseptet fra Anders Midsundstad ettersom det viste seg at han ikke hadde mulighet de to siste ukene. 

\section{Anbefalinger for utvikling og videre arbeid}
\label{section:anbefaling-videre-utvikling}

\subsection{Tekniske avhengigheter}

\paragraph{Feide og tilgang til studentinformasjon}
Feide innlogging vil være optimalt for å lage en trygg tjeneste som er rettet mot studenter. Siden det er studenter ved en høgskole, så har studentene allerede en feide bruker tilgjengelig. Da slipper brukerene å lage seg nye brukere, noe som brukerene som ble testet sa at det ikke var nok insentiv for å gjøre. Dette ble nevnt som en del av moderert brukertesting skisser 1.0, nevnt i delkapittel~\ref{section:Moderert brukertest-skisser-1}.
Gruppen har selv tatt kontakt med Feide, med tanke på deres restriksjoner og sikkerhetskrav. Fra denne samtalen virket det ikke som at de skulle by på noe problemer å bruke feide på den endelige tjenesten. Om tjenesten eventuelt blir fullført av noen andre studenter, er det opp til HIØ om de stoler på at studentene kan lage en tjeneste uten sikkerhetshull. \cite{FEIDE-EPOST:24}
Hvis bruken av feide skulle vise seg å by på problemer å få implementert eller at HIØ ikke kan stole på studentene til å lage en tjeneste uten sikkerhetshull, så må det brukes et annet alternativ for innlogging.
%Her må vi diskutere alternativer til innlogging%

Notater:
Drøfting og vurdering av alternativer til Feide.

\paragraph{Database}
Siden systemet inkluderer innlogging og informasjon om organisasjoner fra andre kilder enn Brønnøysundregisteret vil det være behov for en database som kan lagre informasjon om både brukere, organisasjoner og administrator. Om Feide-innlogging implementeres og dette i tillegg kan brukes til å hente ut informasjon om studenten som logger inn fra HIØ sine datakilder, trengs ikke informasjon om studenter som passord, e-post og navn å lagres i systemets database. Om studenten velger å legge til ytterligere informasjon til profilen sin, som organisasjoner og aktiviteter den er interessert i, beskrivelse eller bosted, vil det være hensiktsmessig å kunne lagre dette i systemets database sammen med en ID knyttet opp mot studenten.

\paragraph{Publiseringsløsning}
Notater:
Vurdere og drøfte ulike publiseringsløsninger: Wordpress eller lignende CMS, plugin eller nettsted, Vortex (HIØ sitt CMS), app, kode eget rammeverk.

\subsection{Utvikling av tjenesten}
Notater:
Vurdering og drøfting av alternativer: bachelorprosjekt, masterprosjekt, outsourcing, delvis outsourcing, samarbeid med kommunen(e).
Kan søke om midler til utvikling.

\subsection{Drift av tjenesten}
Løsninger for enkel drift av tjenesten ble tatt opp som et ønske fra oppdragsgiver i prosjektbeskrivelsen, denne ligger vedlagt i Tillegg~\ref{vedlegg:prosjektbeskrivelse}. Forslaget fra oppdragsgiver lød slik \say{Videre er det ønskelig, basert på kombinasjonen av tilgjengelig informasjon i registeret som eksempelvis navn på organisasjonen og sted, å nyttiggjøre seg av dette for å generere og presentere ytterligere informasjon om organisasjonen til brukeren.} \cite{PROSJEKTBESKRIVELSE:22}. Ettersom oppdragsgiver så for seg å kunne drifte den ferdige tjenesten selv ønsket han å automatisere innhentingen og oppdateringen av informasjon om organisasjoner så mye som mulig. 

På møter med veileder og oppdragsgiver så vi sammen på tekniske muligheter for dette og det ble foreslått å utforske en løsning med en søkerobot (også kalt webcrawler eller spider). Den tekniske gjennomføringen av dette var et fagområde vi i prosjektgruppen ikke var kjent med, så vi oppsøkte Tom Heine Nätt, som er fagansatt ved HIØ med kunnskap innenfor fagfeltet. Han tok opp flere utfordringer ved denne løsningen og anbefalte at om vi ønsket å lage en løsning for å få tak i informasjon om organisasjoner var det nok både enklest og tryggest å la dem legge ut informasjonen selv. \cite{WEBCRAWLER-SAMTALE:23}. Etter å også ha snakket med oppdragsgiver bestemte vi oss for å gå bort fra denne løsningen i prototypen vi skulle utvikle og ta utgangspunkt i forslaget om å la organisasjonene selv legge ut informasjon.

Ettersom den ferdige prototypen ikke inneholder en løsning for å automatisk hente inn informasjon om organisasjoner oppstår det nye utfordringer i forhold til om organisasjoner ville vært aktive i tjenesten. I samtalen vi hadde med Gina Finsrud fra Halden Kommune kommer det fram at dette har vært et problem i et prosjekt hun jobbet med for å samle frivillige organisasjoner på en hjemmeside via Kulturrådet og Idrettsrådet: \say{Organisasjonene drives ofte av frivillige som ikke har tid til å oppdatere siden sin. Kontaktpersoner byttes også ofte og det varierer hvem som er motivert til å være aktiv på nett. Kommunen får penger til å utvikle en tjeneste men ikke til drift, egentlig skulle vi hatt en 50-100\% stilling for å kunne holde tjenesten i gang} \cite{KOMMUNEN-INTERVJU:20}. Det samme problemet ble også tatt opp av Wenche Eriksen på møtet med Halden Frivilligsentral, hun forteller at mange organisasjoner ikke har ressurser til markedsføring og at et slikt system som vi foreslo krever at noen fulgte opp organisasjonene. \cite{FRIVILLIGSENTRALEN-INTERVJU:21}.

Om dette prosjektet skal utvikles videre til et ferdig produkt vil det etter vår mening være nødvendig å ha en god løsning for å skaffe informasjon om organisasjoner og holde denne oppdatert. Derfor skal vi i de neste avsnittene drøfte noen av de mulige løsningene for dette og gi våre anbefalinger til oppdragsgiver.

\subsubsection{Søkerobot}
Det er lett å se fordelene ved å utvikle en søkerobot for å gjøre oppgaven med å finne og oppdatere informasjon om organisasjoner. Om det brukes litt tid og eventuelt penger på å utvikle god funksjonalitet vil dette være en investering fordi mesteparten av driften vil gå av seg selv. Hvis man ser bort ifra å rette opp feil og behandle henvendelser vil det ikke være noe ytterligere driftsbehov i tjenesten ved bruk av denne løsningen. Dette vil da spare både tid og penger til drift, som også kan være vanskelig å oppdrive. Om det skulle bli aktuelt å tilby tjenesten til andre aktører vil det også være enklere for dem å si ja til å ta i bruk et produkt som har lite behov for drift.

Det fins også flere større utfordringer ved denne løsningen. Tom Heine Nätt fortalte at det var ville vært nødvendig å ta høyde for at søkeroboten gjorde feil. Han påpekte at selv om det ikke trengte å være så vanskelig å utvikle en søkerobot som kunne finne informasjon om organisasjoner, så ville det være svært ressurskrevende å utvikle en som kun hentet ut den riktige informasjonen. Om informasjonen som hadde blitt vist til brukeren var feil kunne dette svekke integriteten til både tjenesten og organisasjonene. I tillegg nevnte han også utfordringene med å måtte navigere lover om opphavsrett, andre tjenesters egne retningslinjer og etiske grenser når det kom til å nyttiggjøre seg av informasjon fra andre nettsider og tjenester. \cite{WEBCRAWLER-SAMTALE:23}. For å bygge videre på dette vil det også alltid være en sannsynlighet for at informasjonen som organisasjonene selv har lagt ut ikke er oppdatert eller er feil og dette er ikke noe en søkerobot kan finne ut. Vi tror også at det kan være negativt for organisasjoners tillit til tjenesten om de ser at informasjonen de har lagt ut på sine egne sider har blitt kopiert og brukt i en ny tjeneste, uten at de har blitt kontaktet av en person eller har hatt noen kontroll over dette.

\subsubsection{Opprette egen driftsstilling}
Å opprette en egen stilling i 50-100\% for å drifte tjenesten er den mest nærliggende løsningen for utfordringene tatt opp av både Gina Finsrud og Wenche Eriksen. Fra møtet på Frivilligsentralen ble det tatt opp at kontakt og oppfølging av organisasjoner burde gjøres av en person gjennom fysisk oppmøte eller telefon for at organisasjonene ble mer oppmerksomme på tjenesten. De mente at det var viktig at det var noen med menneskelige kunnskaper som kunne koordinere dette. \cite{FRIVILLIGSENTRALEN-INTERVJU:21}. Vi i prosjektgruppen mener også at det kan være en trygghet for både organisasjoner og studenter å vite at det er en person som styrer, følger med og modererer tjenesten regelmessig ettersom de menneskelige og sosiale aspektene står så sterkt i tjenestens formål. Dette tror vi også vil være en fordel i situasjoner der det er nødvendig å ta egne vurderinger, som om en inaktivprofil skal slettes eller om noe ikke ser riktig ut og burde følges opp. I tillegg hadde en daglig driftsansatt sannsynligvis kjent systemet ut og inn og hadde visst hva som fungerte bra og ikke. Vedkommende hadde dermed enkelt kunne kommet med anbefalinger for videre drift, utvidelse, oppdateringer.

Denne løsningen har også noen åpenbare ulemper. Økonomi er nok den mest sentrale. Ville HIØ betalt for å opprette en stilling for drift av en tjeneste de ikke visste om kom til å fungere? I Halden Kommune møtte Gina Finsrud på hindringen med at det ikke var midler til å drifte en tjeneste \cite{KOMMUNEN-INTERVJU:20}. Da er det også nærliggende å tenke at man ville møtt på det samme problemet ved HIØ. Dessuten må man tenke på hva slags kompetanse en slik stilling måtte hatt. Om stillingen hadde gått ut på å markedsføre opp mot organisasjoner og følge opp dem, i tillegg til å kunne fikse feil og kjenne den tekniske plattformen ut og inn kunne det blitt vanskelig å finne noen som er kvalifisert og i tillegg har lyst til å jobbe innen et nytt og uutprøvd konsept. Arbeidet kunne også blitt kjedelig til tider ettersom den driftsansvarlige hadde måttet bruke mye tid på å søke opp og finne informasjon. Om tjenesten også skulle blitt utvidet og dermed inkludert flere organisasjoner ville man også måttet vurderdere å ansette flere for drift. Om andre aktører på andre steder hadde blitt tilbudt tjenesten, hvem skulle hatt ansvar for oppfølging av deres lokale organisasjoner?

\subsubsection{Halvautomatisering}
Forslaget om halvautomatisering er en hybrid av de to andre. Det vil fortsatt være en person som jobber med driften av tjenesten, men også en søkerobot som bistår i å lete fram informasjon om organisasjoner. For organisasjoner som ikke har opprettet profil enda finner søkeroboten kontaktinfo og presenterer dette til driftsansvarlig, som kan velge å ta kontakt for å invitere organisasjonen til å opprette profil. Slik som i prototypen velger organisasjonene selv å opprette profil i tjenesten, men søkeroboten vil bistå med å finne utfyllende og oppdatert informasjon. Informasjonen funnet av søkeroboten kunne enten ha blitt presentert til driftsansvarlig, som igjen kontakter organisasjonene og spør om de ønsker å legge til eller oppdatere informasjonen. Søkeroboten kunne også presentert det direkte til organisasjonene ved for eksempel å sende ut varsel på e-post, men dette er avhengig av at organisasjonene leser disse. Uansett måtte noen ha godkjent informasjonen før den legges ut på organisasjonenes offentlige side.

Det er mange fordeler med denne løsningen. Det hadde spart tid på drift av tjenesten siden det ikke hadde vært behov for å bruke mye tid på å søke opp informasjon. Dermed kunne stillingen blitt redusert til 20-50\%, avhengig av hvor mye tid man velger å bruke på markedsføring og oppfølging og hvordan det gjennomføres. Sammenlignet med å ha en 50-100\% stilling ville det sannsynligvis vært enklere å få økonomisk gjennomslag for. Det hadde heller ikke nødvendigvis vært behov for å opprette en ny stilling, stillingen kunne blitt gitt til noen som allerede er ansatt, eventuelt fordelt på to ansatte. Etter å ha fått i gang tjenesten i startfasen og fått med brukere hadde driftsansvarliges viktigste rolle vært å dobbeltsjekke om informasjonen er riktig og følge opp organisasjonenes og studentenes aktivitet. Denne løsningen gir altså mest fleksibilitet og valgfrihet både for eier av produktet og for den eller de driftsansvarlige. Dessuten er det en trygghet i at det er personer som jobber med tjenesten og kan ta fortløpende vurderinger om behandling av feil og beslutninger som enten ikke kan eller bør automatiseres.

Denne løsningen har også noen ulemper. Det vil fortsatt koste mer penger enn helautomatisering og en eller flere personer vil fortsatt ha nødt til å bruke en del tid på drift av tjenesten. Dessuten vil de sannsynligvis komme til å ringe til en del feil telefonnummer eller sende e-post til feil adresse, ettersom søkeroboten både kommer til å selv gjøre feil og finne utdatert informasjon. Derfor vil det også være viktig at de som har denne jobben er årvåkne og ikke stoler ukritisk på søkerobotens anbefalinger. Om en søkerobot implementeres vil det fortsatt være nødvendig å ta hensyn til etiske og lovpålagte retningslinjer for bruk av denne. Basert på hvor stor grad av kontakt med organisasjonene driftsstillingen går ut på kan det også oppstå utfordringer om organisasjonene ikke leser varlser på e-post, spesielt om disse er automatisk generert. Dette kan da føre til at organisasjoners bruk av tjenesten dør ut om ikke driftsansvarlig er på banen og følger opp personlig. 

\subsubsection{Prosjektgruppens anbefalinger}
Vår anbefaling er å utvikle tjenesten med forslaget for halvautomatisering som utgangspunkt for løsning av problemene knyttet til drift. Slik som vi ser det har denne løsningen fordelene fra begge de to andre løsningene og færrest ulemper. Ettersom denne løsningen har stor fleksibilitet tror vi også at den vil være mulig å få gjennomslag for. Vi ser for oss at driften vil ta større ressurser i startfasen, da organisasjoner sannsynligvis må kontaktes personlig over telefon eller fysisk oppmøte for å fortelle dem hvorfor de burde bli med. For å gjøre dette effektivt kan dette også gjøres ved å ta kontakt med flere organisasjoner om gangen gjennom for eksempel Idrettsrådet og Kulturrådet og presentere tjenesten til alle. Når tjenesten får økt synlighet vil den da gå inn i en fase med vedlikehold og oppfølging, det er da man tester om tjenesten fungerer i praksis og kan finne ut hva som kreves av driftsstillingen. 

Ved å ha både søkerobot og en driftsstilling på plass vil man ha alle muligheter åpne og det vil ikke kreve så store tiltak for å justere på balansen mellom disse om man ser at driftsaspektet fungerer dårlig. Da vil man også ha målbare effekter å legge til grunn om man ønsker å øke stillingsprosenten eller lignende. Målbare effekter vil også være sentralt å legge fram om man vil tilby tjenesten til andre aktører, da anbefaler vi at det først er oppnådd en metode for drift av tjenesten som fungerer.

\subsection{Merkevarebygging og synliggjøring av tjenesten Aktiv Student}
Konseptet til tjenesten Aktiv Student ble utformet på oppdrag fra HIØ som et av deres bachelorprosjekter. Dette alene kan styrke tjenestens troverdighet blant potensielle brukere og eventuelt andre interesserte aktører, enda mer om HIØ offentlig stiller seg bak tjenesten ved å for eksempel integrere det som en del av sin hjemmeside.

\paragraph{Synliggjøring opp mot studenter}

Tjenesten Aktiv student blir utviklet for nye studenter ved HIØ og skal hjelpe dem i starten av studietiden for å finne fritidsaktiviteter og foreninger å ta del i. Resultater i våre brukertester og undersøkelser viser at dette er noe nåværende studenter ønsker fantes da de startet, og noe kommende studenter sier hadde vært et bra tiltak for dem når de skal studere. 
undersøkelsene våre viser at tjenesten er noe som må introduseres så tidlig som mulig i studietiden. Gruppens ønske av utviklere og administrator er at denne tjenesten må ha en sentral rolle på høgskolens nettside, og bør introduseres i fadderukene samt markedsføres gjennom student politikken og student organisasjoner.

\paragraph{Markedsføring opp mot organisasjoner}

I tiden før en eventuell lansering bør administrator / drift være forberedt på å bidra litt ekstra. Preliminær kontakt med organisasjoner bør bli gjort ved oppsøking og forespørsler. Det er kritisk å vurdere utviklere sin rolle i å automatisere så mye av denne prosessen etterhvert. All jobben som legges ned i oppstarten av Aktiv Student vil være vel verdt det etterhvert når tjenesten blir mer kjent blant organisasjoner å de selv kommer til siden for å registrere seg. Etterhvert når merkevaren blir godt kjent og kanskje har spred seg ved å bli tilbudt eller tatt opp av andre instanser, håper vi at dette vil skape en type bevegelse blant organisasjoner i Norge hvor det å være en del av merkevaren "Aktiv Student" er attraktivt og hjelper til å berike organisasjons samfunnene og nærområdene rundt. Gruppens hovedtips for utviklere og administrator er å starte med studentorganisasjonene å få med alle disse så det vises at HIØ stiller seg bak. Noe som kan hjelpe merkevarebyggningen kraftig er å presentere tjenesten for de forskjellige organisasjons forbundene i Norge å få dem ombord, så de kan markedsføre dette videre til sine organisasjoner. Prøv å skape media dekking rundt merkevaren for å introdusere dette til nærmiljøet. 

\paragraph{Utfordringer}
Utfordringen blir å skape en levende tjeneste, hvor utfyllende informasjon om organisasjoner legges til og ny og oppdatert informasjon blir lagt til kontinuerlig. Dette innebærer at det må være attraktivt for organisasjoner og legge til informasjon og oppdatere sine sider inne på tjenesten. For studenter må det være attraktivt å logge inn eller opprette profil, slik at de kan være med å skape aktivitet i tjenesten. Det vil også bli utfordrende å få tjenesten i gang i startfasen. For at studenter skal ha lyst til å bruke tjenesten må det være organisasjoner der, men om studentene ikke enda har tatt i bruk tjenesten vil nok ikke organisasjonene se noen grunn til å bruke den heller.

Etter startfasen er ferdig og både studenter og organisasjoner har begynt å bruke tjenesten vil det måtte legges ned arbeid for at tjenesten ikke skal dø ut. Man kan ikke regne med at markedsføringen vil gå av seg selv når kull med studenter kommer og drar og kontaktpersoner for organisasjoner byttes ut. Når studenter slutter på skolen vil det sannsynligvis oppstå inaktive profiler som forstatt står i oversikten over {\em interesserte studenter} på organisasjonssidene. Om dette ikke ryddes opp i kan det bli et problem for denne funksjonens troverdighet om det blir en kjent sak at antallet interesserte studenter ikke stemmer. Inaktive organisasjonsprofiler kan også bli et problem. Om kontaktpersoner byttes men den nye ikke har fått innloggingen til tjenesten vil det kunne føre til at organisasjonsprofilen står fast med utdatert info uten at en kontaktperson svarer interesserte studenter i chat. Opprydding av inaktive profiler vil derfor være en viktig oppgave for driftsansvarlig.

Etter møtet med Halden Frivilligsentral ble vi gjort oppmerksom på en annen utfordring. Wenche Eriksen fortalte at mange av organisasjonene i Halden er allerede mettet. Flere organisasjoner har venteliste og for noen av dem må man kjenne noen og få en anbefaling for å bli med. Dette skaper problemer for tilflyttende studenter, som vanligvis ikke kjenner noen fra før i organisasjonen. Hun hadde også erfart skiller i samfunnet som hindret personer som ikke tilhørte riktig sosiale gruppe å bli medlem i organisasjoner. Det er viktig at organisasjonene ikke forskjellsbehandler og at de kan være møtesteder for alle. \cite{FRIVILLIGSENTRALEN-INTERVJU:21}


\paragraph{Hvordan kan dette løses?}
Å få de første brukerne inn i tjenesten vil sannsynligvis ikke skje automatisk. For å få ballen til å rulle vil vi derfor anbefale å be om starthjelp et utvalg organisasjoner, her kan for eksempel organisasjonene under Studentsamfunnene og andre studentdrevne grupper være lett å involvere. Om man får disse med som de første organisasjonene i tjenesten vil man da ha noe å tilby studentene i markedsføringen opp mot dem. I boken {\em Interactive Design: An Introduction to the Theory and Application of User-centered Design} sier forfatteren at \say{Users can be motivated in multiple ways—through rewards, through a clear understanding of what an application offers them, and through delivering what they want in the fastest, best way possible.} \cite[123]{INTERACTIVE-DESIGN:19}. Gjennom å fortelle studentorganisasjonene hva de kan få ut av tjenesten vil de motiveres til å hjelpe til i startfasen, dermed har man noe å vise studentene slik at de ser konkret hva de kan få ut av tjenesten. Om studentene da begynner å bruke tjenesten vil den kunne levere et fritidstilbud til dem med én gang. Om dette da øker synligheten og engasjementet for organisasjonene vil dette bli deres belønning. Som en ekstra belønning for organisasjoner som er aktive og oppdaterer informasjonen sin så kan det introduseres et system som plasserer aktive organisasjoner lengre opp på resultatlistene i tjenesten. Etter å ha sett at tjenesten fungerer i praksis vil det da bli aktuelt å markedsføre til større grupper med organisasjoner og studenter. 

Den endelige tjenesten bør være godt synlig og lett tilgjengelig plassert på forsiden av høgskolens nettsted \footnote{https://www.hiof.no/} for at det skal bli brukt av studenter. Dette vil fungere som en passiv form for markedsføring. Når det kommer nye kull med studenter sjekker de ofte høgskolens nettsted og blir dermed introdusert til tjenesten. Det vil også være aktuelt å drive med kontinuerlig aktiv markedsføring og synliggjøring av tjenesten for å holde den i live. Den som er ansvarlig for dette burde ha oversikt over når det kommer nye studenter og når man enklest kan nå disse, men også når organisasjoner har årsmøte og kanskje bytter styremedlemmer og kontaktperson. I tillegg vil det være en fordel å ha oversikt over arrangementer og begivenheter som kan benyttes som treffpunkter og vil være ekstra aktuelle å synliggjøre seg på. Ved å ha oversikt over dette vil det kunne lages en årsplan for markedsføring med oversikt over når man burde markedsføre til hvem, hvordan man burde markedsføre til de ulike målgruppene og hvilke kanaler man burde bruke.

Om det oppstår problemer med lange ventelister for organisasjoner eller forskjellsbehandling kan det være en løsning å legge mer til rette for at studenter skal kunne opprette egne organisasjoner i tjenesten. Slik som den ferdige prototypen er utformet så er dette fullt mulig, men det er ikke noen ytterligere tilrettelegging eller veiledning til studenter om hvordan man gjør dette. Om dette skal gjøres må det tydeliggjøres for personene som oppretter organisasjonene hva det innebærer og hvilket ansvar de tar på seg. 
Wenche Eriksen forteller at det ofte er tilfeller der personer som ønsker å starte opp grupper men plutselig mister motivasjonen. Det er viktig at de som driver organisasjonene skjønner hva frivillighet går ut på, de må ha noe å tilby for å kunne få tilbake engasjement og felleskap fra medlemmene sine og være fort på banen når noen vil bli med. Hun advarer mot å gjøre det for lett å opprette en organisasjonsprofil i tjenesten, i fare for at de ender opp som inaktive profiler. Hun legger fram et scenario der en person som har sosiale vansker har bygget opp mot i lang tid for å ta kontakt med en organisasjon de er interessert i, men når de tar kontakt så blir de avvist fordi organisasjonen ikke er aktiv lengre. Dette kan føre til at de ikke tør å prøve å kontakte andre organisasjoner og at man mister dem helt. \cite{FRIVILLIGSENTRALEN-INTERVJU:21}


\subsection{Mulighet for utvidelse av tjenesten}
Som en del av oppdraget ble tjenesten utviklet i hovedsak for HIØ. Men etter oppdragsgivers ønske er også muligheter for utvidelse av tjenesten til bruk for andre aktører utforsket. Informasjonen om lag, foreninger og organisasjoner som ligger tilgjengelig på Brønnøysundregisterets API dekker hele Norge. Det betyr at systemet kan enkelt få tilgang på all denne informasjonen. Ved noen enkle endringer i parameterene i systemets kode kan tjenesten tilbys til for eksempel andre høyskoler, kommuner, fylkeskommuner og andre interesserte aktører. Ved bruk av aktører utenfor sektor for høyere utdanning vil Feide-innlogging måtte erstattes eller suppleres med andre løsninger, det vil være aktuelt å undersøke løsninger som for eksempel MinID \footnote{http://eid.difi.no/nb/minid} eller BankID \footnote{https://www.bankid.no/bedrift/}. Ved å tilby tjenesten til større aktører kan dette også bidra til å skape et større navn for tjenesten, som igjen kan bidra til å synliggjøre tjenesten og skape mer tillit overfor organisasjoner og privatbrukere, øke antall brukere og skape en god sirkel av merkevarebygging for alle parter.

\paragraph{Løsninger for kommersialisering}

Mulighetene for utvidlese og kommersialisering rundt den endelige tjenesten har stort potensiale. Som vi allerede har nevnt tidligere, så anbefales det å ha en fast stilling på 20-50 \% så driften opprettholdes. Men om dette skal markedsføres og selges videre til andre høgskoler, komunner, forbund og instanser vil vi anbefale og øke stillings prosenten til 100 eller leie inn markedsførings personell. Om dette skal utvides til andre instanser så krever det også en endring av innlogging metode som ikke bare er for studenter, her vil vi anbefale enten bank-ID eller minID for å ivareta sikkerheten til brukerene. Hvis høgskolen ser at det er stor interesse rundt tjenesten vil dette være spesielt viktig å ha en talls/kontaktperson for tjenesten som da vil kreve større arbeidskraft. Noe høgskolen kan gjøre for å skaffe midler og investering i tjenesten vil være å prøve å appellerer til kommunen for midler ved å trekke frem samfunnsnytten det vil gi dem ved å ha en slik tjeneste i nærområdet. Noe som også kan vurderes av produkteier er det å slå seg sammen med kommunen i et partnerskap under utviklingen av produktet ettersom de kan få støtte til å skape en slik tjeneste.

Et av våre gruppemedlem hadde en prat med Gina Finsrud i Halden Kommune hvor hun kom med noen viktige punkter som vil være forutsetninger for at den endelige tjenesten skal kunne utvides. Hun sa at kontinuerlig markedsføring, penger til drift, investering av tid og kontakt utover med organisasjoner er veldig viktig for at denne tjensten skal bli en suksess.\cite{KOMMUNEN-INTERVJU:20} 

Når det kommer til problemet med hvordan skal man få finansiert utviklingen og ikke minst driften av en slik tjeneste så har vi et par forslag. Man kan gå inn i et samarbeid med kommunen, samle penger vi kickstarter, få private investorer ombord eller andre utdanningsinstitusjoner og til slutt søke om bidrag for folkehelsefremmende/samfunnsbyggende arbeid.




\paragraph{Markedsføring opp mot potensielle kunder eller investorer}
Notater:
Folkehelse, mer effektiv i arbeid, samfunnsbygging, tilhørighet til kommunene, studenter melder flytting, bryr seg mer om stedet de bor og bidrar i samfunnet, frivillighet får alt til å gå rundt, kompetansetilflyttere får tilknytning og blir.

Studier viser oss at det å være i fysisk aktivitet eller det å ha en hobby bidrar til blant annet: bedret folkehelse, mer effektivitet i arbeidslivet, økt selvfølelse av tilhørlighet,bedre samfunns velferdsel og bedre personlig selvfølelse. For å få denne tjenesten ut i liv, som noe som kan bidra til disse punktene blant samfunn som kommuner, skoler og andre nærmiljø er det viktig at det blir markedsført på en måte som drar frem disse mulige verdiene. Når dette blir pitchet til potensielle kunder og investorer er det viktig å vise at denne tjeneste ikke bare tilbyr et tjeneste som oppretter bedre kommunikasjon mellom samfunn og organisasjoner, men at det muligens kan dra frem disse verdiene å føre til et bedre samfunn i sin helhet.


\section{Konklusjon og oppsummering}
Coming soon to a hoveddokument near you






