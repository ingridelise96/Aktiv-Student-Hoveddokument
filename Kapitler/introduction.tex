\cleardoublepage
\chapter{Introduksjon}
\label{chap:intro}

\section{Prosjektgruppen}
%-----------Sørge for å dele opp de 4 gruppemedlemmene med litt luft.--------------%
Prosjektgruppen er sammensatt av fire studenter ved IT-avdelingen ved Høgskolen i Østfold, tre fra Digitale Medier og Design og én student fra Informatikk - Design og utvikling av IT-systemer.

Ingrid Elise Dahl studerer Informatikk - Design og utvikling av IT-systemer, bor i Halden og er fra Trondheim. Ingrids faglige interesser innebærer frontend-utvikling, design, datasikkerhet, databasesystemer og applikasjonsutvikling.

Markus Arnø Madsen studerer Digitale medier og Design, bor i Fredrikstad og er fra Trondheim. Markus sin faglige interesse innebærer design, prosjektledelse, interaksjons design, brukerorientert design, markedsføring og prosjekt/produkt utvikling.

Anders Walle Pettersen studerer Digitale medier og Design, bor i Fredrikstad. Anders sin faglige interesse innebærer design, interaksjons design, spillutvikling, grafisk design og brukerorientert design.

Stefan Larsen studerer Digitale medier og Design, og bor i Halden. 3D og Grafisk design, samt Webutvikling har vært noen av de mest interessante veiene å utforske gjennom studietiden på HIOF. 


\section{Oppdragsgiver}
Gruppens oppdragsgiver er Tommy Payne på vegne av Høgskolen i Østfold (omtalt som HIØ videre i dokumentet). HIØ har 7700 studenter og 620 ansatte fordelt på to campus i Halden og Fredrikstad\footnote{https://www.hiof.no/om/}. Det er fem fagavdelinger, ett akademi og ett senter ved HIØ\footnote{https://www.hiof.no/om/organisasjon/fagavdelinger/}.

Tommy Payne jobber som Seniorkonsulent i Studieenheten ved HIØ og inngår i team for Studieutredning og kvalitetssikring med ansvar for det helhetlige læringsmiljøet ved campus Halden og campus Fredrikstad. Arbeidet hans går ut på å sikre og videreutvikle det digitale, fysiske, organisatoriske og psykososiale læringsmiljøet ved høgskolen. \footnote{https://www.hiof.no/om/organisasjon/administrasjonen/organisasjons-og-tjenesteutvikling/studenttjenester/personer/tekn-adm-ansatte/tommypa/}

\section{Oppdraget}
Løsningen som gruppen har kommet fram til i samråd med oppdragsgiver er en prototype av en nettbasert plattform for studenter der de kan finne organisasjoner, lag og foreninger i sitt nærområde. Prosjektet baserer seg på at gruppen skal lage denne prototypen og hoveddokumentet, hvor disse sammen skal fungere som en manual/bruksanvisning for en annen gruppe studenter som senere kan videreføre produktet til ferdig stand.  Plattformen bruker informasjon fra Brønnøysundregisteret som grunnlag, men oppfordrer i tillegg organisasjoner til å registrere informasjon om seg selv. Prosjektgruppen vil kun utvikle en prototype av denne plattformen, men vil beskrive hva sluttresultatet burde være og hva som skal til for å oppnå dette. 

Det er sentralt at plattformen er tilstrekkelig gjennomført samt godt markedsført at organisasjonene selv oppsøker den for å registrere seg. I oppstartsfasen vil det være nødvendig å ta kontakt med organisasjoner direkte for å oppfordre dem til å registrere seg.

Oppdragsgiver Tommy Payne jobber med læringsmiljø og studentenes trivsel og helse ved HIØ. Å utvikle en plattform som hjelper studenter å bli mer aktive, delta mer sosialt og engasjere seg i lokalmiljøet vil være av interesse for både studentene, Tommy Payne, HIØ som institusjon og nærmiljøet rundt Høgskolen.

\section{Problemstilling}
Studenter ved HIØ har pr. i dag ikke tilfredsstillende verktøy for å finne oversikt over fritidsaktiviteter og ta kontakt med organisasjoner som arrangerer disse. Hvordan kan man designe et verktøy som legger til rette for at studenter enklere kan delta på fritidsaktiviteter og ta kontakt med organisasjoner, lag og foreninger i sitt nærområde? Hvordan kan tjenesten utformes på en måte som gjør det interessant og attraktivt for studenter å gjøre dette? Hvordan kan denne tjenesten synliggjøres for både studenter og organisasjoner, lag og foreninger med den hensikt at den skal bli brukt av flest mulig i målgruppen?

\subsection{Utvikling av hypoteser}
\label{section:hypoteser}
Prosjektgruppen utviklet fire hypoteser som i løpet av prosjektet ble utforsket gjennom brukerintervjuer -og undersøkelser, forskning og fagpersoner. I utviklingen av hypotesene la prosjektgruppen vekt på det sosiale og samfunnsmessige aspektet i tillegg til det tekniske aspektet. Den viktigste grunnen til dette var at det sosiale og samfunnsmessige aspektet med tjenesten var et viktig fokusområde for oppdragsgiver og derfor burde prosjektgruppens funn innen dette området ligge til grunn for tekniske og designrelaterte beslutninger tatt under utformingen av tjenesten.

Hypotesene fungerte som ledetråder for prosjektgruppen i arbeidet med å utforme produktet. I brukerundersøkelsene ble spørsmålene og samtaletemaene vinklet mot temaene i hypotesene ettersom hypotesene oppsummerte prosjektgruppens tanker og antakelser om hva som var viktigst å legge vekt på i tjenesten. Om noen av disse antakelsene i løpet av undersøkelsene ble motbevist eller fikk liten støtte ville det føre til at tema i brukerundersøkelsene måtte vinkles annerledes.

\paragraph{Hypoteser}
\begin{compactitem}
\item[{\bf H1}] En tjeneste som tilrettelegger for å delta sammen med andre likesinnede kan senke terskelen for å delta på en aktivitet
\item[{\bf H2}] Studenter ved HIØ syns det er vanskelig å finne en oversikt over aktivitetstilbud
\item[{\bf H3}] Studenter ved HIØ syns terskelen for å selv ta kontakt med organisasjoner og aktivitetsgrupper er for høy 
\item[{\bf H4}] Spennende funksjoner og godt design skaper en tjeneste som studenter ved HIØ ønsker å benytte
\end{compactitem}

\section{Formål, leveranser og metode}
\label{sec:maal-metode-resultater}

\subsection{Formål}

\begin{compactitem}
\item [{\bf Hovedmål}] Det kommer fram av Studentenes Helse -og Trivselsundersøkelse 2018 at nesten hver tredje student opplever ensomhet i studietiden. \cite{SHOT:2} Målet med plattformen vil være å skape en løsning for studenter slik at de enklere kan få oversikt over og komme i kontakt med frivillige organiasjoner, lag og foreninger i sitt nærområde. Samt skape en bedre og enklere løsning enn den som finnes idag. Dette vil igjen berike nærmiljøet og føre til mindre ensomhet blant studenter.
\begin{compactitem}
\item [{\bf  Delmål 1: Studentaspektet} ] Gjøre det enklere for studenter ved HIØ å finne og komme i kontakt med organisasjoner i nærområdet og andre studenter med like interesser. Øke studentengasjementet og skape lavterskel muligheter for sosialisering og fellesskap for alle studenter ved HIØ.
\item [{\bf  Delmål 2: Organisasjonsaspektet} ] Gjøre det enklere for organisasjoner å bli sett av studenter ved HIØ for å øke medlemstall og engasjement i organisasjonene.
\item [{\bf  Delmål 3: Samfunnsaspektet} ] Skape kontaktpunkter mellom studenter og fastboende i studiekommunene. Skape tilhørighet for studenter i sin studiekommune, som igjen kan øke engasjementet og motivasjonen til å bidra i sitt nærområde.
\item [{\bf  Delmål 4: Synlighetsaspektet} ] Skape et synlig produkt som alle studenter ved HIØ og organisasjoner i nærområdet vet eksisterer og vet hvor de kan finne. Skape et produkt som studenter og organisasjoner selv ønsker å bruke og ser nytten ved dette.
\end{compactitem} 
\end{compactitem}

\section{Analyse}
Her blir forskjellige aspekter av oppgaven beskrevet og det er gjort opp noen tanker om gjennomføringen. Det er også en analyse av markedet der det fokuseres på merkevarebygging av tjenesten Aktiv Student, det er gjort en heuristisk analyse og en benchmarking av eksisterende arbeid og tilbud. Relatert arbeid blir også beskrevet og sammenlignet opp mot prosjektoppgaven.

\subsection{Målgruppen}
I oppgavebeskrivelsen gitt av oppdragsgiver er målgruppen satt til å være studenter ved HIØ. Videre skal Aktiv Student kunne utvides til å omfatte målgrupper ved andre utdanningsinstitusjoner i Norge. 
% --------Her bør vi kanskje skille litt bedre fra student og organisasjon.-----------%
Ettersom nesten hver tredje student opplever ensomhet ifølge Studentenes Helse -og Trivselsundersøkelse \cite{SHOT:2} vil Aktiv Student være spesielt relevant, siden målet med plattformen er å gjøre organisasjoner, lag og foreninger mer tilgjengelige for studenter.

Plattformen skal brukes av organisasjoner, lag og foreninger i området rundt HIØ, disse vil også være en del av målgruppen. Hvilke av disse organisasjonene som er mest relevante å inkludere i plattformen skal undersøkes gjennom brukerundersøkelser.

\subsection{Relaterte plattformer}
%-------------Den første setningen herunder trenger en kildehenvisning.--------------%
Mennesker trenger hverandre for å overleve, og viktigheten i det å knytte mennesker sammen blir stadig diskutert. Det er mye som tilsier at det blir vanskeligere å føle en tilhørighet til andre etter hvert som hverdagene blir mer stressfulle og digitale. Man må også tenke på de som lider av ensomhet eller økonomisk utrygghet, samt mennesker med fysiske og psykiske lidelser.

\vspace{5mm} %5mm vertical space
Nasjonal kompetansetjeneste for barn og unge med funksjonsnedsettelser har en metode kalt Fritid med Bistand.\footnote{\url{https://www.fritidmedbistand.no/om-metoden.326149.no.html}} Der oppfordres kommuner til å gjennomføre metoden som etablerer en støttekontakt for å hjelpe den enkelte med å finne inkludering i en eller flere fritidsorganisasjoner. Anders Midtsundstad\footnote{\url{http://www.mynewsdesk.com/no/valnesfjord-helsesportssenter/contact_people/90351}},  
 forfatteren bak metoden samt seniorrådgiver og prosjektleder ved Nasjonal kompetansetjeneste for barn og unge med funksjonsnedsettelser, fortalte via en innledende samtale at man må etablere et tillitsforhold, slik at når døren til aktiviteten er synlig åpen, så blir det desto lettere å gå igjennom den. Disse dørene kan bli åpnet av støttekontakter, faddere eller øvrige frivillige. 

\vspace{5mm} %5mm vertical space
Flere har tenkt på sosial inkludering, og har prøvd å komme med gode løsninger. Meetup.com er en nettside for å knytte mennesker med felles interesser, med et søkelys på å lære nye ting sammen.\footnote{https://www.meetup.com/} Ideen kom fra kontaktsøkende mennesker etter at tragedien inntraff i New York i 2001.\footnote{https://archive.triblive.com/lifestyles/more-lifestyles/pittsburgh-meetup-members-use-the-internet-to-get-off-the-internet/}

Nextdoor.com fokuserer på forholdet til naboene dine.\footnote{https://www.nextdoor.co.uk/} Det er ikke så vanlig å bli kjent med naboene sine lengre, men vi har fortsatt behov for å av og til låne hekkesakser, stiger eller kanskje en gressklipper. Ved å kreve autentisering av bostedsadresse sørger de for at man kun forholder seg til de som bor rundt deg. Sistnevnte er dog kun aktuell for mennesker bosatt i England.

MeetMe\footnote{https://www.meetme.com/} og Skout\footnote{https://www.skout.com/} kan sees på som varianter av Tinder, applikasjoner for møte nye mennesker i ditt nærområde, som blir definert gjennom fellestrekk i brukerprofilene.

\subsubsection{Heuristisk analyse}

\paragraph{Forklaring}

For å bedre kunne forstå brukeropplevelsen på Aktiv Student, kan prosjektteamet selv teste og evaluere bestemte funksjoner før videre arbeid. Det er viktig at funksjonene som testes er oppgaver som den typiske bruker er nødt til å gjennomføre for å oppnå det han eller hun ønsker.
\linebreak 
Under følger 3 oppgaver som er tilpasset brukervennlighet. Hver oppgave er blitt vurdert etter følgende heuristikker, hentet fra Jakob Nielsens ti generelle prinsipper for interaksjonsdesign. \footnote{https://www.nngroup.com/articles/ten-usability-heuristics/}

 \begin{itemize}
    \item \textbf{Match mellom system og den virkelige verden}   %line with dot
          \say{Systemet skal bruke språk som er enkelt for målbrukeren å forstå.} Vi vil finne og luke ut ord som studenter og unge finner vanskelige eller uforståelige.   %line without dot 
    \item \textbf{Brukerstyring og frihet} 
            \say{Brukere skal kunne angre og gjøre om handlinger, og også ha muligheten til å bevege seg bakover. Det skal alltid være en nødutgang.} Gmail lar deg \say{angre} handlinger. Dette er viktig siden folk gjør feil, og trenger evnen til å rette opp disse feilene.
    \item \textbf{Konsistens og standarder}  %line with dot
          \say{Skjermelementer skal være konsistente over hele plattformen.}
            Microsofts topplinjer er identiske for Word, Excel og Powerpoint, slik at brukere kan bytte mellom de tre applikasjonene sømløst.   %line without dot 
    \item \textbf{Fleksibilitet og effektivitet i bruk} 
            Snarveier lar ekspertbrukere få fart på opplevelsen, samtidig som de blir usett for den nybegynneren. Disse snarveiene er viktige fordi systemet ditt skal imøtekomme alle fagområder.
\end{itemize}


\subsubsection{Hvordan bedømmes det}

Det finnes flere grader alvorlighet ved funn av feil, og for å kunne bedømme mest mulig konsistent så er det blitt brukt en 0 – 4 gradering skala, hvor 4 blir regnet som en katastrofal feil. Tabell ~\ref{table:heuristikkGrader} beskriver alvorlighetsgradene av feil.
\footnote{https://www.nngroup.com/articles/how-to-rate-the-severity-of-usability-problems/}


\begin{center}
\begin{table}[H]
\begin{tabular}{ c l }
 \textbf{Grad} & \textbf{Beskrivelse}  \\
 0 & Dette vil ikke bli ansett som en brukervennlig feil   \\
 1 & Kun kosmetisk feil. \\
 2 & Mindre brukervennlig problem. Bør gis lav prioritet.  \\
 3 & Stort brukervennlig problem. Bør gis høy prioritet  \\
 4 & Katastrofal feil i brukervennlighet. Avgjørende å rette opp.
\end{tabular}
 \caption{Liste over alvorlighetsgrader ved feil}
 \label{table:heuristikkGrader}
\end{table}
\end{center}


Heuristikkene vil bli utført på 3 aktuelle nettsider. HIOF sine egne sider for fritidstilbud, Brønnøysund sine sider for fritidstilbud, samt Meetup sine gruppesider. Vi vil vise til hvor hvert tilfelle har funnet sted.
\vspace{5mm} %5mm vertical space
\newpage
\textbf{Oppgave 1} \par
Angre på valg av fritidstilbud.


Bruker klikker seg inn på helt inn på et valgfritt fritidstilbud. \newline Ved landing av side, angre avgjørelsen og velg et annet fritidstilbud.\newline
Dette er et mikrosenario som kan forekomme ved feilklikk, anger eller ombestemmelse av bruker. Tabell ~\ref{table:angreTilbud} viser en undersøkelse av dette scenariet.
% ----------Her bør vi få fontstørrelsen større inne i tabellene.----------- %
\begin{center}
\begin{table}[H]
\begin{tabular}{ | m{0.7cm} | m{4cm}| m{4cm} | m{4cm} | } 
 \hline
 \centering Grad & Lokasjon & Beskrivelse & Anbefaling \\
 \centering 3 & \tiny https://www.hiof.no/livet-rundt-studiene/studentforeninger/lag-og-foreninger/ & \tiny Ikke eksisterende funksjon for å angre. Kun tilbakeknapp i nettleser sender bruker tilbake. & 
 \tiny 
 \begin{itemize} 
    \item Ved filtrering av søk, legg ved \say{X} i fanen, for å umiddelbart rette opp.
\item Åpne eksterne sider i nye faner.
\end{itemize}  \\
 \centering 3 & \tiny https://www.hiof.no/livet-rundt-studiene/studentforeninger/lag-og-foreninger/ & \tiny Hver gang man åpner et nytt fritidstilbud forlater man hiof.no og lander på en facebook gruppe side. & \tiny Å Integrere mest mulig kontaktinformasjon på hiof sine eksisterende sider. Legg heller ved URL til facebook etc.  \\
 \centering 3 & \tiny https://w2.brreg.no/frivillighetsregisteret/ & \tiny Ikke eksisterende funksjon for å angre. Må bruke tilbakeknapp i nettleser eller \say{Nytt Søk} knappen for å gjøre nytt søk. & \tiny Ved filtrering av søk, legg ved \say{X} i fanen, for å umiddelbart rette opp. \\
\centering 2 & \tiny https://www.meetup.com/ & \tiny Ikke-eksisterende angre funksjon. Istedenfor har Meetup en klikkbar link til arrangøren, som sender bruker til evt andre arrangementer fra samme arrangør. Meetup leverer også en \say{Lignende arrangementer i nærheten} fane nedenunder valgt arrangement.  & \tiny Ved filtrering av søk, legg ved \say{X} i fanen, for å umiddelbart rette opp. \\
 \hline
\end{tabular}
\caption{Undersøkelse av scenariet \say{angre på valg av fritidstilbud} ved forskjellige plattformer}
\label{table:angreTilbud}
\end{table}
\end{center}

\textbf{Oppgave 2} \par
Bruk eksisterende filter for å kategorisere et søk av fritidstilbud


Bruker benytter eksisterende metoder for filtrering av søk for å finne ett eller flere fritidstilbud.\newline
Eksempel: Sport / Idrett - Ballspill - Fotball.
\par
Dette er et scenario som vil forekomme av brukere som velger å ikke bruke fritekstsøk av ulike grunner. Hensiktsmessig hvis bruker ikke husker navnet på forening eller om bruker leter etter noe nytt å utforske. Tabell ~\ref{table:brukFilter} viser en undersøkelse av dette scenariet.

\begin{center}
\begin{table}[H]
\begin{tabular}{ | m{0.7cm} | m{4cm}| m{4cm} | m{4cm} | } 
 \hline
 \centering Grad & Lokasjon & Beskrivelse & Anbefaling \\
 
 \centering 4 & \tiny https://www.hiof.no/livet-rundt-studiene/studentforeninger/lag-og-foreninger/ & \tiny Det finnes ingen mulighet for filtering og kategorisering. Kun utvalgte URL som er direkte tilknyttet HIOF sine tilbud. & \tiny \begin{itemize}
    \item Implementer søkemotor som lar brukere kategorisere innhold etter ønsker og behov.
\item Inkludere fritidstilbud i nærområdet som er ikke assosiert med skolen.\end{itemize}  \\

 \centering 2 & \tiny https://w2.brreg.no/frivillighetsregisteret/ & \tiny Søketreff oppdateres ikke før bruker har trykket søk. & \tiny Lag en løsning slik at etterhvert som kriteriene blir satt, så vil alle gyldige tilbud bli vist under søkefeltet.  \\
 
 \centering 3 & \tiny https://w2.brreg.no/frivillighetsregisteret/ & \tiny Unødvendige trinn for filtrering av søk. & \tiny Felter som \say{Landsdel} kan sløyfes. Det er også et felt om \say{Fylke}, så det skal være nok. \\
 
 \centering 2 & \tiny https://w2.brreg.no/frivillighetsregisteret/ & \tiny Uforståelige kategorinavn i \say{Annet} & \tiny Implementer en ryddigere inndeling av kategorier så man slipper å benytte seg av slikt. \\
 
 \centering 3 & \tiny https://w2.brreg.no/frivillighetsregisteret/ & \tiny Uforståelige kategorinavn i \say{Legater og Fremme av Frivillighet} & \tiny Bruk studenter til å finne egne navn som er lettere å forstå. \\
 
  \centering 2 & \tiny https://www.meetup.com/welcome/categories/ & \tiny Ved første gangs innlogging bes bruker å velge en eller flere favorittkategorier som et utgangspunkt. Dette skjer i form av store, fargerike ikoner. Man kommer ikke videre før man har valgt noen. & \tiny Vurdering om denne funksjonen er nødvendig eller evt valgfri, via en synlig link fra de øvrige sidene. \\
 \hline
\end{tabular}
\caption{Undersøkelse av scenariet \say{bruk eksisterende filter for å kategorisere et søk av fritidstilbud} ved forskjellige plattformer}
\label{table:brukFilter}
\end{table}
\end{center}
\newpage 
\textbf{Oppgave 3} \par
Oppsøk grunnleggende kontaktinformasjon til ulike fritidstilbud / foreninger.


Bruker klikker seg inn på helt inn på ett eller flere fritidstilbud. Finn navn, adresser, telefonnumre og nettsider til disse.

Dette er et scenario som vil forekomme når bruker ønsker å ta kontakt med vertene av fritidstilbudene / foreningene.
Tabell ~\ref{table:oppsokKontaktinfo} viser en undersøkelse av dette scenariet.

\begin{center}
\begin{table}[H]
\begin{tabular}{ | m{0.7cm} | m{4cm}| m{4cm} | m{4cm} | } 
 \hline
 \centering Grad & Lokasjon & Beskrivelse & Anbefaling \\

 \centering 4 & \tiny https://www.hiof.no/livet-rundt-studiene/studentforeninger/lag-og-foreninger/ & \tiny Ofte ikke eksisterende. Alle foreninger blir vist på sine facebook sider, og noen av gruppene er også kun for medlemmer. & \tiny Vis grunnleggende kontaktinformasjon på hiof.no sine sider, og link heller til facebook for de med behov.  \\
 
 \centering 4 & \tiny https://w2.brreg.no/frivillighetsregisteret/ & \tiny Uoversiktlig vei inn til kontaktinformasjon. Må trykke på organisasjonsnummeret i listen over tilbud for å se. & \tiny Gjør det mer synlig hvor man må trykke for å se kontaktinformasjon. \\
 
 \centering 3 & \tiny https://w2.brreg.no/frivillighetsregisteret/ & \tiny Unødvendig informasjon vises. & \tiny Ikke prioriter å vise alt av informasjon til studenten. \\
 
 \centering 4 & \tiny https://w2.brreg.no/frivillighetsregisteret/ & \tiny Manglende informasjon vises. & \tiny Sørg for at alle har en viss mengde informasjon om seg. (tlf, e-post, adresse etc). \\
 
  \centering 2 & \tiny https://www.meetup.com/ & \tiny Man vises en tekstblokk med informasjon om tilbudet, samt en sidebar med navn på arrangør, adresse og tidspunkt. Under vises et google-kart av lokasjonen. Man kan også se hvilke deltakere som har meldt seg på. & \tiny Foruten navn, adresse og tidspunkt, prøve å inkludere mest mulig informasjon som også et telefonnummer og en e-postadresse. \\
 \hline
\end{tabular}
\caption{Undersøkelse av scenariet \say{oppsøk grunnleggende kontaktinformasjon til ulike fritidstilbud / foreninger} ved forskjellige plattformer}
\label{table:oppsokKontaktinfo}
\end{table}
\end{center}

\subsection{Benchmarking}

\paragraph{Brreg.no:}
Dette er nettsiden vi skal gå ut i fra når vi planlegger og setter opp vårt prosjekt.
Siden er ganske uoversiktlig, spesielt når det kommer til menyen.
Designet på siden er simpelt, noe kjedelig og se på.
Selve informasjonen på siden er litt uoversiktlig, i tillegg får man presentert den i søketreff veldig tett på hverandre i alfabetisk rekkefølge, dette gjør det vanskelig å skille informasjon og finne det man er ute etter som bruker.

\paragraph{Frivilligsentral.no:}
Dette er en søkeportal med oversikt over frivillighetsorganisasjoner i Norge. De har gjort det noe bedre en Brreg.no når det kommer til oversiktlighet med tanke på at de har markert hvert enkelt treff i søkeresultatlisten sin men her gis det også ut kun det aller mest nødvendige av informasjon og ikke mer. Selve designet her er litt mer lekent og fargerikt, men de bruker lys grønn farge på en del av teksten sin som kan bli veldig vanskelig å lese for mange.

\paragraph{Finn.no}
Dette nettstedet er jo ikke en direkte konkurrent opp mot vårt nettsted, men kan være greit å analysere pga deres søkemotor som vil gi oss en bedre oversikt over hva vi kan ha med på vår. For eksempel dette med flere kategorier mangler på våre nåværende konkurrenter, i tillegg kanskje ta noen ideer fra fremvisning av informasjon og struktur på siden.

\paragraph{Meetup.com}
Oversatt fra engelsk:
\say{Meetup er en plattform for å finne og bygge lokalsamfunn. Mennesker bruker Meetup for å møte nye mennesker, lære nye ting, søke støtte, komme seg ut av komfortsonen, og forfølge sine lidenskaper, sammen.}  \footnote{https://www.meetup.com/about/} Primært en amerikansk tjeneste, men med noen få grupper arrangert i Norge.


\paragraph{Nextdoor.co.uk}
Oversatt fra engelsk:
\say{Nextdoor er nabolagets knutepunkt for pålitelige nettverk og behjelpelig informasjon, varer og tjenester. Vi tror at ved å bringe naboer sammen, så kan vi dyrke en vennligere verden hvor alle har et nabolag de kan stole på.}. Denne tjenesten er utelukket alle utenfor England, men konseptet er svært interessant \footnote{\url{https://about.nextdoor.com/gb/\#our-manifesto}} .

% --------------------- Dette må fullføres ----------------------------- %
\subsection{Relaterte studier og publikasjoner (uferdig)}

\paragraph{Deltakelse i gruppeaktiviteter}
Det har blitt skrevet om digitale plattformer som oppfordrer brukere til å delta på gruppeaktiviteter i artiklene ''Group-Activity Organizing Through an Awareness-of-Others Interface'' \cite{AWARENESS:3:CSCW18} og ''To Go or not to Go!: What Influences Newcomers of Hybrid Communities to Participate Offline'' \cite{NEWCOMERS:4:CT17}. ''Group-Activity Organizing Through an Awareness-of-Others Interface'' tar utgangspunkt i en app som lar brukere organisere aktiviteter og andre brukere delta på disse. Fokuset ligger på å gjøre organiseringen og gjennomføringen av aktiviteter så lett som mulig. ''To Go or not to Go!: What Influences Newcomers of Hybrid Communities to Participate Offline'' tar utgangspunkt i plattformen Meetup.com og undersøker hvilke faktorer som spiller inn for at brukere skal ta steget til å møte opp på en aktivitet for første gang.

\paragraph{Kontakte ukjente}
Det er også skrevet artikler om å få ukjente til å ta kontakt med hverandre på digitale plattformer. ''Outlining the design space of playful interactions between nearby strangers'' \cite{NEARBY:5:AM16} beskriver en designprosess med workshops for å utvikle ideer for hvordan fremmede mennesker som befinner seg i nærheten av hverandre kan samhandle på forskjellige måter. ''Playfulness and progression in technology-enhanced social experiences between nearby strangers'' \cite{PLAYFUL:6:NORDICHI18} beskriver videreføringen av denne prosessen, der en app som heter Next2You blir utviklet etter prinsippene fra design-workshops og testet i et bruker-studie. Appen lar brukerne automatisk utveksle valgfri informasjon med andre brukere som kommer innen mobiltelefonens Bluetooth-radius.

\paragraph{Forslag til aktiviteter}
Anbefaling av aktiviteter har blitt skrevet om i artiklene ''A Novel Method for Event Recommendation in Meetup'' \cite{MEETUP:7:ASONAM17} og ''Users psychological profiles for leisure activity recommendation: user study'' \cite{PROFILES:10:CITREC17}. Begge disse artiklene samler inn store mengder informasjon om brukere som brukes til å beregne hvilke aktiviteter som kan være interessante og foreslå disse for brukeren.

\paragraph{Matchmaking av brukere}
Aspektet med matchmaking av brukere er skrevet om i ''Connecting users with similar interests via tag network inference'' \cite{TAGNETWORK:8:CIKM11}, som bruker tagger i forskjellige sosiale nettverk og blogger for å finne likheter mellom brukere. I ''Matchmaking in p2p e-marketplaces: soft constraints and compromise matching'' \cite{MATCHMAKING:9:ICEC10} samles det også inn informasjon om brukeren og det brukes en algoritme for å finne brukere som burde ta kontakt med hverandre.

\paragraph{Fritid med bistand}
Publikasjonene ''Fritidsorganisasjoner vil og de kan'' \cite{FRITID:12}, ''Inkludering gjennom organiserte fritidsaktiviteter'' \cite{INKLUDERING:11} og ''Tillit, mestring og selvoppfatning'' \cite{TILLIT:13} har fokus på aktivisering hos personer med nedsatt funksjonsevne, lav inntekt eller andre tydelige hindringer som gjør at de kan trenge støtte for å delta på aktiviteter. Disse publikasjonene er relatert til Anders Midtsundstad sin metode ''Fritid med Bistand'' \footnote{https://www.fritidmedbistand.no/}. 
% ---------- Annen tittel enn Nye aspekter etc?---------------%
\paragraph{Nye aspekter jobbet med i prosjektoppgaven}
Mange av ideéne og prinsippene som er beskrevet i disse artiklene tas med videre av prosjektgruppen i arbeidet med plattformen. Det er flere aspekter av dette bachelorprosjektet som ikke er beskrevet i publikasjonene nevnt ovenfor. En av disse er autonomi hos brukeren. Et viktig poeng med oppdraget er at brukeren selv skal ville dele informasjon, ta kontakt og delta på aktiviteter. Det vil både være best for brukeren og enklest for utviklingsteamet om brukeren selv velger hvilken informasjon som skal deles. Et annet aspekt som er lite sett på er involveringen av allerede eksisterende organisasjoner og hvordan brukerens deltakelse kan berike både organisasjonene og nærmiljøet sitt. 

I tillegg er det slik at de fleste studiene fokuserer enten på brukere som allerede er sosiale og ikke har problemer med å ta kontakt med andre, eller brukere som har ulike hindringer eller funksjonsnedsettelser som øker terskelen for at de vil ta kontakt eller delta. Ved at målgruppen i dette oppdraget er først og fremst alle studenter ved HIØ vil plattformen i dette bachelorprosjektet ta hensyn til alle typer mennesker med forskjellige grader av sosiale preferanser, forskjellige motivasjoner, og også personer med forskjellige typer nedsatt funksjonsevne.

\section{Rapportstruktur}
\paragraph{Kapittel 2} inneholder en beskrivelse av arbeidsprosessen med billedlig og skriftlig dokumentasjon, brukerundersøkelser, utvikling av skisser, brukertesting, evalueringer og forbedringer som er gjort.

\paragraph{Kapittel 3}
inneholder en beskrivelse av det det ferdige resultatet som gruppen produserte og om det oppnådde krav og spesifikasjoner satt på forhånd.

\paragraph{Kapittel 4}
diskuterer resultatet og måloppnåelsen. Her vil det foreligge en vurdering av arbeidet og anbefalinger og designprinsipper for videre arbeid. Gruppen vil også diskutere erfaringer gjort underveis i arbeidet. Avslutningsvis legges frem konklusjon og oppsummering.

