\cleardoublepage
\chapter{Resultater}
\label{chap:results} 

Dette kapittelet inneholder en presentasjon av produktet gruppen har produsert. Her evalueres arbeidet, selve produksjonen og hvordan dette ble gjennomført, hvordan resultatet ble og om det oppnådde krav og spesifikasjoner satt på forhånd.

\section{Presentasjon av resultat}

\section{Resultater fra undersøkelse av hypoteser}

De fire hypotesene som prosjektgruppen utarbeidet i begynnelsen av prosjektet ble undersøkt gjennom gruppens egne brukerundersøkelser, konsultasjon av fagpersoner og relevant ekstern forskning. Dette ble gjort parallelt med arbeidet med produktet nettopp fordi hypotesene oppsummerte antakelser og spørsmål som prosjektgruppen ønsket å få svar på underveis i prosessen for å kunne utarbeide et godt produkt.

\subsection{H1: Å delta sammen med andre likesinnede senker terskelen for å delta på en aktivitet}
\paragraph{Brukerintervjuer}
Hypotese 1 fikk støtte allerede under de initielle brukerintervjuene, beskrevet i delkapittel~\ref{section:init-brukerintervjuer}. 8/8 deltakere av brukerintervjuene svarte at å få med en venn eller bekjent ville gjort det enklere for dem å bli med på en organisert fritidsaktivitet. Dessuten svarte 7/8 deltakere at de tidligere hadde deltatt på en fritidsaktivitet fordi venner eller bekjente var med. De fleste av disse svarte også at det sosiale fellesskapet var en viktig grunn til at de valgte å delta. Noen svarte også at de ikke ville deltatt om de ikke hadde kjent til noen andre som deltok på aktiviteten.

\paragraph{Invitasjon til deltakelse gjennom mellomledd}
7/8 deltakere av de initielle brukerintervjuene svarte at gitt at de hadde en bekjent som var medlem i en organisasjon de var interesserte i, ville de tatt kontakt med den bekjente for å finne ut mer om organisasjonen. På den andre siden ville kun 2/8 deltakere tatt direkte kontakt med organisasjonen. Dette bygger opp under antakelsen om at personer i målgruppen heller ønsker å tilnærme seg organisasjonen gjennom et mellomledd enn å ta kontakt og delta alene. 

Dette ble gitt videre støtte av fagperson Anders Midtsundstad, som har utviklet metoden Fritid med Bistand \footnote{https://www.fritidmedbistand.no/}. Midtsundstad skrev i en samtale med prosjektgruppen via e-post at \say{I metoden Fritid med Bistand er det saksbehandler som har et ansvar i arbeidet med metoden til å legge til rette for deltakerne. Dette innebærer behov for å bruke den tid som kreves for å etablere tillit mellom metodens aktører. [...] I metoden Fritid med Bistand handler det om unge og eldre med ulike former for bistandsbehov. Studentene som er deres målgruppe har ofte en annen situasjon, men tillitsrelasjonen må en alltid forholde seg til. Det betyr i praksis at en tredjeperson bør invitere til fellesskapet.} \cite{MIDTSUNDSTAD-EPOST:14}

\paragraph{Konklusjon}
Etter undersøkelse av hypotese 1 kom prosjektgruppen fram til konklusjonen at å delta sammen med andre eller å bli invitert til å delta av et mellomledd øker det sosiale og menneskelige aspektet ved aktiviteten, noe som kan føre til at studenten raskere føler en tilhørighet, får større tillit til organisasjonen og blir mer motivert til å delta. Ergo bekreftes hypotese 1.

\subsection{H2: Studenter ved HIØ syns det er vanskelig å finne en oversikt over aktivitetstilbud}

\subsection{H3: Studenter ved HIØ syns terskelen for å selv ta kontakt med organisasjoner og aktivitetsgrupper blir for høy }

\subsection{H4: Spennende funksjoner og godt design skaper en nettside som studenter ved HIØ vil bruke}


\section{Testing og evaluering}

\subsection{Teknisk funksjonalitet}
%Testing av skissenes funksjonalitet på 3 deltakere (Ingrid)%

\subsection{Studenters vurdering}
%Spørsmål stilt til studenter%

\subsection{Organisasjoners vurdering}
%Spørsmål til organisasjoner%

\subsection{Fagpersoners vurdering}
%Skrive om tidligere mening om konseptet fra møtet med Midtsundstad. Han hadde ikke tid til å gi en grundigere vurdering. Vurdering fra NAV-ansatt?%

\subsection{Oppdragsgivers vurdering}
%Får vurdering på møtet 22/5%



