\cleardoublepage
\chapter{Resultater}
\label{chap:results} 

Dette kapittelet inneholder en presentasjon av produktet gruppen har utviklet. Her beskrives også gjennomførte evalueringer av arbeidet, selve produksjonen og hvordan dette ble gjennomført, hvordan resultatet ble og om det oppnådde krav og spesifikasjoner satt på forhånd.

\section{Presentasjon av produktet}

\section{Resultater fra undersøkelse av hypoteser}

De fire hypotesene som prosjektgruppen utarbeidet i begynnelsen av prosjektet ble undersøkt gjennom gruppens egne brukerundersøkelser, konsultasjon av fagpersoner og relevant ekstern forskning. Dette ble gjort parallelt med arbeidet med produktet nettopp fordi hypotesene oppsummerte antakelser og spørsmål som prosjektgruppen ønsket å få svar på underveis i prosessen for å kunne utarbeide et godt produkt.

En sikker bekreftelse eller avkreftelse av hypotesene var enten ikke mulig eller ikke praktisk gjennomførbart for prosjektgruppen, men dette var heller ikke hensikten med hypotesene. Hypotesene ble brukt som en ledetråd for å finne de viktigste fokusområdene i videre undersøkelser og dermed i design av tjenesten. Det var derfor viktigst for prosjektgruppen å vite om funn fra undersøkelsene styrket eller svekket hypotesene, slik at fokusområdene eventuelt kunne justeres etter som.

\subsection{H1: En tjeneste som tilrettelegger for å delta sammen med andre likesinnede kan senke terskelen for å delta på en aktivitet}

\paragraph{Brukerintervjuer}
Hypotese 1 fikk støtte allerede under de initielle brukerintervjuene, beskrevet i delkapittel~\ref{section:init-brukerintervjuer}. 8/8 deltakere av brukerintervjuene svarte at å få med en venn eller bekjent ville gjort det enklere for dem å bli med på en organisert fritidsaktivitet. Dessuten svarte 7/8 deltakere at de tidligere hadde deltatt på en fritidsaktivitet fordi venner eller bekjente var med. De fleste av disse svarte også at det sosiale fellesskapet var en viktig grunn til at de valgte å delta. Noen svarte også at de ikke ville deltatt om de ikke hadde kjent til noen andre som deltok på aktiviteten.

\paragraph{Invitasjon til deltakelse gjennom mellomledd}
7/8 deltakere av de initielle brukerintervjuene svarte at gitt at de hadde en bekjent som var medlem i en organisasjon de var interesserte i, ville de tatt kontakt med den bekjente for å finne ut mer om organisasjonen. På den andre siden ville kun 2/8 deltakere tatt direkte kontakt med organisasjonen. Dette bygger opp under antakelsen om at personer i målgruppen heller ønsker å tilnærme seg organisasjonen gjennom et mellomledd enn å ta kontakt og delta alene. 

Dette ble gitt videre støtte av fagperson Anders Midtsundstad, som har utviklet metoden Fritid med Bistand \footnote{https://www.fritidmedbistand.no/}. Midtsundstad skrev i en samtale med prosjektgruppen via e-post at \say{I metoden Fritid med Bistand er det saksbehandler som har et ansvar i arbeidet med metoden til å legge til rette for deltakerne. Dette innebærer behov for å bruke den tid som kreves for å etablere tillit mellom metodens aktører. [...] I metoden Fritid med Bistand handler det om unge og eldre med ulike former for bistandsbehov. Studentene som er deres målgruppe har ofte en annen situasjon, men tillitsrelasjonen må en alltid forholde seg til. Det betyr i praksis at en tredjeperson bør invitere til fellesskapet} \cite{MIDTSUNDSTAD-EPOST:14}. 

Intervjuet med kvinnen som jobber i NAV indikerer mye av det samme. Kvinnen sa i intervjuet at \say{det kan virke som mange må ha noen som kan fysisk følge dem til arrangementer, ei hånd å holde i. Det hjelper ikke hvor mange tilbud man får når man er alene} \cite{NAV-INTERVJU:16}. Her var altså hjelp fra et mellomledd og deltakelse sammen med noen andre sentralt for at personene skulle tørre å dra på arrangementer. 

\paragraph{Konklusjon}
Etter undersøkelse av hypotese 1 kom prosjektgruppen fram til at ved å tilrettelegge i tjenesten for å delta sammen med andre eller å bli invitert til å delta av et mellomledd er det sannsynlig at dette kan føre til at studenter blir mer motivert til å delta. Gjennom at det sosiale og menneskelige aspektet ved tjenesten blir gjort tydelig kan studenten få større tillit til organisasjonen \cite{MIDTSUNDSTAD-EPOST:14}. Uten å utvikle tjenesten og teste denne i bruk er det dog umulig å si sikkert om tjenesten faktisk vil senke terskelen for å delta på aktiviteter, men prosjektgruppens funn styrker hypotese 1.

\subsection{H2: Studenter ved HIØ syns det er vanskelig å finne en oversikt over aktivitetstilbud}

\paragraph{Brukerintervjuer}
Svar fra de initielle brukerintervjuene støttet opp under hypotese 2. 6/8 deltakere svarte at å få bedre oversikt og tilgang til aktivitetstilbud ville gjøre det enklere for dem å delta på en aktivitet. Det som samtidig ble nevnt var vanskeligheten med å finne fram til kontaktinformasjon eller informasjon om hvordan man kunne delta på en aktivitet. 7/8 deltakere svarte at lett tilgjengelig kontaktinformasjon til organisasjoner spilte en viktig rolle i om de ville ta kontakt eller ikke. I tillegg svarte 8/8 deltakere at de ville tatt kontakt med en interessant organisasjon om det hadde eksistert gode tjenester som gjorde det enkelt å ta kontakt.

Deltakerne som ikke hadde deltatt på en organisert fritidsaktivitet i løpet av studietiden ble spurt om hva som holdt dem fra å delta. Én deltaker svarte \say{det er for dårlig tilbud. Jeg har sjekket tilbudet til HSS men ingenting der fenger.} En annen deltaker sa at han \say{vet ikke hva som er tilgjengelig og da gidder jeg ikke å begynne å lete.} Det ble også nevnt av flere deltakere at det var viktig for dem å ha lett tilgang til god og oppdatert informasjon om organisasjonene, ettersom flere av deltakerne heller ville lest om dem og møtt direkte opp på møte enn å tatt kontakt med organsasjonen først.

\paragraph{Mangler ved dagens aktivitetstilbud}
I alle rundene med brukerundersøkelser som ble gjennomført ble deltakerne spurt om sine tanker om konseptet Aktiv Student. Svarene som kom fram pekte på en mangel på et oversiktlig aktivitetstilbud for studenter ved HIØ i dag. I førsteinntykkstesten av skisser 1.0 sa en deltaker at han \say{liker konseptet, det er mangel på dette i dag og studenter ved høgskolen trenger det}. I brukertesten av skisser 2.0 ble en ny gruppe med deltakere spurt samme spørsmål. Det ble sagt av en deltaker at plattformen \say{svarer på et eksisterende behov ved høgskolen.} En annen deltaker sa at han ville brukt plattformen fordi \say{informasjon om organisasjoner på HIØ i dag er overfladisk og spredt overalt}.

Et viktig element i alle brukerundersøkelsene var at deltakerne skulle kunne snakke fritt og med så få føringer fra prosjektgruppen som mulig. Problemene med aktivitetstilbudet ble følgelig ble tatt opp på eget initiativ av studentene som deltok i undersøkelsene. Av mange deltakere ble denne problemstillingen også nevnt som første tema i samtalene om fritidsaktivitetstilbudet for studenter ved HIØ.

\paragraph{Konklusjon}
Etter å ha snakket med til sammen 13 forskjellige studenter ved HIØ, noen av dem i flere omganger, forelå det mange svar og bemerkninger som styrket antakelsen om at studenter syns at dagens aktivitetstilbud ved HIØ er mangelfullt og uoversiktlig. Samtidig kom det ingen svar som svekket antakelsen. Det faktum at deltakere tok opp problemene med tilbudet uten direkte oppfordring fra prosjektgruppen ga antakelsen ekstra tyngde. Ettersom prosjektgruppen kun hadde tilgang på en liten testgruppe kunne ikke hypotesen testes grundig nok til å kunne bekreftes, men undersøkelsene som ble gjennomført styrker hypotese 2.

\subsection{H3: Studenter ved HIØ syns terskelen for å selv ta kontakt med organisasjoner og aktivitetsgrupper er for høy}

\paragraph{Brukerintervjuer}
Svar fra de initielle brukerintervjuene ga blandende tilbakemeldinger. 6/8 deltakere svarte at det hadde gjort det enklere for dem å delta på en organisert fritidsaktivitet om det hadde vært lettere å ta kontakt eller om organisasjonen hadde initiert kontakt. I tillegg svarte kun 2/8 deltakere at de ville tatt kontakt om de hadde funnet en organisasjon de var interessert i. Dette så ut til å peke mot at deltakerne syntes terskelen var for høy til å ta kontakt, men på den andre siden var det flere som svarte at de heller ville ha deltatt direkte på en aktivitet med organisasjonen uten å trenge å ta kontakt først. Dette indikerte at det ikke var {\em for høy terskel} som var problemet for deltakerne, ettersom å delta på en aktivitet er et steg videre fra å ta kontakt.

\paragraph{Fagpersoner og forskning}
I intervjuet med kvinnen som jobbet med sosialhjelp i NAV ble det tatt opp flere punkter som var relevante for hypotese 3. Det ble tatt opp at flere av NAV-brukerne syntes det generelt var vanskelig å initiere kontakt, også med personer som jobbet i NAV. Hun opplevde ofte at hun måtte ta den første kontakten med brukerne fordi mange hadde {\em telefon-angst} og slet med å ta kontakt selv \cite{NAV-INTERVJU:16}.

Funn fra forskningsartikkelen {\em To Go or not to Go!: What Influences Newcomers of Hybrid Communities to Participate Offline} indikerte at en av årsakene til at brukere av tjenesten Meetup \footnote{https://www.meetup.com/} valgte å ikke delta på en aktivitet kunne være den sosiale distansen mellom aktivitetens vert og brukeren. Undersøkelsene i artikkelen kom blant annet fram til at om verten var en erfaren bruker av tjenesten var det mindre sjanse for at en nykommer ville delta på arrangementet. \say{Host tenure was represented as the log of number of days since joining the group; i.e. about 3 (2.7) extra days of group membership for event host in the group results to a decrease of 23\% in likelihood of a newcomer choosing the event as their first one to attend.}. En av anbefalingene nevnt i artikkelen for å motvirke brukerens følelse av sosial avstand var å tilby direkte kontakt med aktivitetens vert for brukeren. \cite{NEWCOMERS:4:CT17}

\paragraph{Konklusjon}
Undersøkelsene av hypotese 3 pekte mot to forskjellige årsaker til at personer valgte å ikke ta kontakt med en organisasjon. En mulig årsak kunne være både at terskelen av ulike grunner var for høy, slik intervjuet med kvinnen fra NAV og funn fra forskningsartikkelen indikerte. En annen mulig årsak kunne være at de ikke så hensikten med å ta kontakt når de heller kunne møte opp direkte på møtet, som funn fra initielle brukerintervjuer indikerte. Prosjektgruppen valgte derfor å designe funksjonalitet med hensyn til begge disse mulige årsakene. Ettersom årsaken til at studenter ikke ville ta kontakt med organisasjoner var uklar og testgruppen samtidig var for liten for å trekke konklusjoner førte undersøkelsene til at hypotese 3 ble svekket. Fokuset ble skiftet fra {\em kontakt-aspektet} til å {\em legge til rette for at studenters mulighet for deltakelse i organisasjoner skulle være enkel og ha lav terskel.}


\subsection{H4: Spennende funksjoner og godt design skaper en tjeneste som studenter ved HIØ vil bruke}


\section{Testing og evaluering}
%Hva ville vi få ut av testingen? Vite om produktet oppfylte krav og forventninger. Tips til videre utvikling og markedsføring%

\subsection{Teknisk funksjonalitet}
%Testing av skissenes funksjonalitet på 3 deltakere (Ingrid)%

\subsection{Studenters vurdering}
% Skriv litt om hensikten med undersøkelsen og gjennomføringen %
% beskrivelse av testdeltakere: kjønn, alder, bakgrunn, studiesituasjon, andre ting? %
% sitat pakkes inn i \say{sitat her} %
Deltaker 1 er en mann som bor ved og studerer ved Campus Horten. Han studerer en master i Maritim ledelse og gikk 1.året når testen ble utført.

Deltaker 2 er en mann som bor på kråkerøy og studerer Økonomi og administrasjon og gikk 2.året når testen ble utført. %var det slik du tenkte at det skulle bli skrevet?%
%Spørsmål stilt til studenter%
\paragraph{1. Finner du konseptet bak dette produktet interessant og brukbart og er dette noe du ville selv brukt?}
%oppsummering av svar fra alle%
Deltakerene syntes at konseptet hørtes interessant ut og tror det kan bli nyttig for mange studenter. De fleste deltakerene ville prøvd plattformen om det hadde vært tilgjengelig på deres studiested, mens en deltakere var litt usikker på om de ville brukt plattformen.
%bare gjør om for det som må endres så det passer til deres svar også.%

\subsubsection{2. Hvordan ser du for deg at dette produktet fungerer i praksis?}

{\bf a. Ville det blitt brukt?}
Så lenge produktet blir markedsført og introdusert for nye studenter, samtidig som det er lett å bruke for både studenter og organisasjoner. Det er også viktig at aktuelle og aktive organisasjoner blir registrert og synligjort på plattformen. Når dette er gjort riktig tror deltakerene at plattformen ville blitt brukt. 

{\bf b. Ville det hjulpet nye studenter?}
Deltakerene mener at denne plattformen ville kunne hjelpe studenter som ikke har noe forhold til nærområde, studenter som vil bli medlem av et miljø og studenter som kan synes at det er litt vanskelig å komme i kontakt med andre.

{\bf c. Tror du det ville blitt brukt til sin hensikt?}
Siden platformen er til hjelp for både studenter og organisasjoner så tror deltakerene at plattformen kommer til å bli brukt til sin hensikt.

\subsubsection{3. Hva tror du samfunnsverdien og nytten ville vært av et slikt produkt i forhold til.}

{\bf a. Relasjon mellom studenter og fastboende i kommunen}
Dette kan være en god mulighet for fastboende og studenter til å bygge større nettverk og bli kjent der de bor under studietiden.
{\bf b. Tilhørighet til plassen man bor}
Studenter kan få det lettere med å føle en tilhørighet til område de bor i hvis de deltar i aktiviteter med andre som bor der.
{\bf c. Psykisk helse og ensomhet}
Denne plattformen kan være med på å hjelpe til å styrke psykisk helse blandt studenter ved at de blir kjent med andre, kan drive med aktiviteter og generelt redusere ensomheten.
\paragraph{Hvordan kunne dette blitt synliggjort og markedsført best mulig opp mot studenter slik at de får lyt til å bruke produktet?} Det blir viktig å introdusere plattformen til studentene så tidlig som mulig, f.eks i fadderuken eller med mail fra skolen. Plakater og eventuelt informasjonsskjermer rundt på skolen ville også være en mulighet.

\paragraph{Er det noe du ville endret med konseptet eller måten det fungerer på?} Deltakerene klarte enten ikke komme på noe de ville endret eller syntes at plattformen virket som den var et godt gjennomtenkt og enkelt konsept å ta i bruk slik det ble presentert.

\paragraph{Har du noen tanker om produktet eller konseptet du føler er relevant å ta med? (positivt eller negativt)}
For at dette skal fungere optimalt, er det viktig å ta kontakt med organisasjonene og får de til å aktivt bruke plattformen og få med så mange som mulig. Hvis studenter blir med og ser at det er få organisasjoner eller at de ikke er aktive, vil dette fort resultere i at produktet ikke blir tatt i bruk senere og kan få et dårlig rykte. 
\subsection{Organisasjoners vurdering}
%Spørsmål til organisasjoner%

\subsection{Fagpersoners vurdering}
%Skrive om tidligere mening om konseptet fra møtet med Midtsundstad. Han hadde ikke tid til å gi en grundigere vurdering. Vurdering fra NAV-ansatt?%

\subsection{Oppdragsgivers vurdering}
%Får vurdering på møtet 22/5%



