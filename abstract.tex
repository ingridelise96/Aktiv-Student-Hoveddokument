\cleardoublepage

\pagenumbering{roman} \setcounter{page}{1}
\chapter*{Sammendrag}
Gjennom bruk av metodikk knyttet til fagområdene Informasjonsarkitektur og Brukerorientert Design skal det utformes en digital løsning som kan benyttes av studenter ved HIØ for å få oversikt over fritidstilbudet innen lokale lag, foreninger og organisasjoner (også omtalt kun som {\em organisasjoner} videre i dokumentet). Prosjektets samfunnsmessige bakgrunn og menneskelige aspekt skal benyttes som retningslinjer under utformingen av løsningen med formål om å skape et lavterskel-tilbud som studenter ved HIØ ønsker å benytte seg av.
I metode kapittelet presenteres og dokumenteres prosessen med å komme frem til en prototype ved bruk av en iterativ Design Thinking-metodikk \footnote{https://www.interaction-design.org/literature/topics/design-thinking}. Kapittelet beskriver brukerintervjuer, undersøkelser og tester av personer i målgruppen, prosessen med å skissere og designe prototypen, samt retningslinjer, fagkunnskaper og forskning som ligger til grunn for valg gruppen har tatt i designet av tjenesten.

Resultat kapittelet inneholder en presentasjon av prototypen gruppen har utviklet. Her beskrives gjennomførte evalueringer av arbeidet, selve produksjonen og hvordan dette ble gjennomført, hvordan resultatet ble,  det presenteres tester og vurderinger av prototypen og konseptet fra studenter, organisasjoner, fagpersoner og oppdragsgiver.

Diskusjons kapittelet vil fokusere på om resultatet ble som forventet, om oppdragsgiver var fornøyd med resultatet, om gruppen kunne ha gjort noe annerledes/bedre og hva gruppen har lært underveis. Det vil også inneholde en vurdering av tjenesten Aktiv Student opp mot relatert arbeid. Gruppen vil gi anbefalinger og fremlegge designprinsipper for videre arbeid og utvikling av tjenesten i praksis. Kapitellet avsluttes med en konklusjon med en oppsummering av arbeidet og prosessen.